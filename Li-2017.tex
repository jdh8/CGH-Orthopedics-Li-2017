\documentclass{beamer}
\usepackage[no-math]{fontspec}
\usepackage{xeCJK}
\usepackage{wasysym}
\setCJKmainfont{Source Han Sans TW}
\hypersetup{colorlinks,linkcolor=}

\usetheme{CambridgeUS}
\title[Refeacture of OVCF s/p kyphoplasty]{
    Refracture of osteoporotic vertebral body after treatment by balloon
    kyphoplasty: Three cases report
}
\subtitle{Xigong Li, Yang Lu, Xiangjin Lin, \textit{Medicine}, 2017}
\author[Chen-Pang He]{何震邦 (Chen-Pang He), Intern}
\date{March 11, 2019}
\institute[CGH]{Cathay General Hospital}

\newcommand*{\solo}[1]{\centering\includegraphics[width=\textwidth, height=0.8\textheight, keepaspectratio]{#1}}

\begin{document}
\maketitle

\section{Introduction}
\begin{frame}{Balloon kyphoplasty}
    Balloon kyphoplasty has been widely accepted as an effective, minimally
    invasive treatment of symptomatic osteoporotic vertebral compression
    fracture (OVCF).

    \begin{itemize}
        \item Significant pain relief
        \item Functional improvement
        \item Vertebral height restoration
    \end{itemize}
\end{frame}

\begin{frame}{Complications}
    \begin{itemize}
        \item Bleeding
        \item Infection
        \item Cement extravation and embolism
        \item Adjacent level fractures following the kyphoplasty procedure
    \end{itemize}
\end{frame}

\begin{frame}{Refracture}
    \begin{itemize}
        \item Refracture of previously treated vertebra is relatively rare.
        \item Very few articles have focused on refracture.
        \item Several predisposing factors and potential mechanisms of
              refracture have been sporadically mentioned in previous case
              reports.
    \end{itemize}
\end{frame}

\begin{frame}{Patient concerns in this study}
    Cases with refracture of OVCF s/p kyphoplasty

    \begin{itemize}
        \item 3 from the First Affiliated Hospital of Zhejiang University
        \item Related published literatures
    \end{itemize}
\end{frame}

\section{Case 1, 86 \male}
\begin{frame}{Case 1}
    \begin{itemize}
        \item 86 y/o man with T12 OVCF
        \item kyphoplasty using polymethylmethacrylate (PMMA) cement via a
              bipedicular approach at a local hospital
        \item Severe osteoporosis (T-score $-4.0$)
    \end{itemize}
\end{frame}

\begin{frame}{}
    \solo{F1.jpg}
\end{frame}

\section{Case 2, 82 \male}
\begin{frame}{}
    \solo{F2.jpg}
\end{frame}

\section{Case 3, 75 \female}
\begin{frame}{}
    \solo{F3.jpg}
\end{frame}

\section{Discussion}
\begin{frame}{Discussion}
    \begin{itemize}
        \item The phenomenon of refracture of the cemented vertebrae after
              kypholasty has been sporadically reported in recent literatures.
        \item Only 14 cases have been reported, including the present 3 cases.
    \end{itemize}
\end{frame}

\begin{frame}{Patient characteristics}
    \solo{T1.pdf}
\end{frame}

\begin{frame}{Patient characteristics}
    \begin{itemize}
        \item All patients obtained pain relieve immediately after kyphoplasty,
              but developed refractures of the previously operated vertebrae at
              the meantime of 3 months (range, 1--4 months) postoperatively.
            \begin{itemize}
                \item Mechanical strength of the augmented vertebrae could not
                      be restored sufficiently by kyphoplasty in the short term.
            \end{itemize}
        \item Vast majority of OVCFs (12/14) had the presence of intravertebral
              clefts in preoperative images.
    \end{itemize}
\end{frame}

\begin{frame}{}
    \solo{F4.jpg}
\end{frame}

\end{document}
