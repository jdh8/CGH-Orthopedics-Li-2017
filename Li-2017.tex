\documentclass{beamer}
\usepackage[no-math]{fontspec}
\usepackage{xeCJK}
\usepackage{wasysym}
\setCJKmainfont{Source Han Sans TW}
\hypersetup{colorlinks,linkcolor=}

\usetheme{CambridgeUS}
\title[Refeacture of OVCF s/p kyphoplasty]{
    Refracture of osteoporotic vertebral body after treatment by balloon
    kyphoplasty: Three cases report
}
\subtitle{Xigong Li, Yang Lu, Xiangjin Lin, \textit{Medicine}, 2017}
\author[Chen-Pang He]{何震邦 (Chen-Pang He), Intern}
\date{March 11, 2019}
\institute[CGH]{Cathay General Hospital}

\newcommand*{\solo}[1]{\centering\includegraphics[width=\textwidth, height=0.8\textheight, keepaspectratio]{#1}}

\begin{document}
\maketitle

\section{Introduction}
\begin{frame}{Balloon kyphoplasty}
    Balloon kyphoplasty has been widely accepted as an effective, minimally
    invasive treatment of symptomatic osteoporotic vertebral compression
    fracture (OVCF).

    \begin{itemize}
        \item Significant pain relief
        \item Functional improvement
        \item Vertebral height restoration
    \end{itemize}
\end{frame}

\begin{frame}{Complications}
    \begin{itemize}
        \item Bleeding
        \item Infection
        \item Cement extravation and embolism
        \item Adjacent level fractures following the kyphoplasty procedure
    \end{itemize}
\end{frame}

\begin{frame}{Refracture}
    \begin{itemize}
        \item Refracture of previously treated vertebra is relatively rare.
        \item Very few articles have focused on refracture.
        \item Several predisposing factors and potential mechanisms of
              refracture have been sporadically mentioned in previous case
              reports.
    \end{itemize}
\end{frame}

\begin{frame}{Patient concerns in this study}
    Cases with refracture of OVCF s/p kyphoplasty

    \begin{itemize}
        \item 3 from the First Affiliated Hospital of Zhejiang University
        \item Related published literatures
    \end{itemize}
\end{frame}

\section{Case 1, 86 \male}
\begin{frame}{Case 1}
    \begin{itemize}
        \item 86 y/o man with T12 OVCF
        \item Severe osteoporosis (T-score $-4.0$)
        \item kyphoplasty using polymethylmethacrylate (PMMA) cement via a
              bipedicular approach at a local hospital
    \end{itemize}
\end{frame}

\begin{frame}{Plain films before and after kyphoplasty}
    \solo{F1a.jpg}
\end{frame}

\begin{frame}{First F/U}
    \begin{itemize}
        \item VAS improved from 9 to 3.
        \item The patient continued to have back pain in the same area 2 weeks later.
    \end{itemize}
\end{frame}

\begin{frame}{Unfilling intraosseous cleft (T2)}
    \solo{F1c.jpg}
\end{frame}

\begin{frame}{Condensed cement and slightly shortened T12}
    \solo{F1d.jpg}
    \footnotetext{Slight back pain relief was attained after 1-month conservative treatment.}
\end{frame}

\begin{frame}{Loss of F/U}
    \begin{itemize}
        \item The patient did not adhere to further follow-up examination and
              oral antiosteoporotic medications, until he experienced a sudden,
              severe onset of unrelenting back pain more severe than any
              previous episodes 10 month later.
        \item His VAS pain score was 9 points.
        \item He refused surgical therapy but elected and adhered to previous
              conservative treatment.
    \end{itemize}
\end{frame}

\begin{frame}{Refracture and 16-month F/U}
    \solo{F1e.jpg}
\end{frame}

\section{Case 2, 82 \male}
\begin{frame}{Case 2}
    \begin{itemize}
        \item 82 y/o man with L3 OVCF
        \item Severe osteoporosis (T-score $-3.8$)
        \item Unilateral transpedicular kyphoplasty
        \item VAS 10 $\to$ 3
    \end{itemize}
\end{frame}

\begin{frame}{Pre-, post-op, and refracture}
    \solo{F2a.jpg}
\end{frame}

\begin{frame}{L3 refracture and new L2 OVCF}
    \solo{F2d.jpg}
\end{frame}

\begin{frame}{Second kyphoplasty}
    \solo{F2f.jpg}
\end{frame}

\begin{frame}{Final outcome}
    \begin{itemize}
        \item The operation was successful and the VAS pain score improved from 10 to 4.
        \item Unfortunately, the patient died of pulmonary infection 3 month later.
    \end{itemize}
\end{frame}

\section{Case 3, 75 \female}
\begin{frame}{Case 3}
    \begin{itemize}
        \item A 75 y/o woman received surgical decompression and instrumented
              fusion due to chronic low back pain and neurological claudication
              caused by degenerative scoliosis.
        \item Osteoporosis (T-score $-3.0$)
        \item Sudden-onset severe back pain 4 months after the primary operation
        \item L1 OVCF $\to$ bilateral transpedicular kyphoplasty
        \item VAS 10 $\to$ 2
    \end{itemize}
\end{frame}

\begin{frame}{Before and after primary operation}
    \solo{F3a.jpg} 
\end{frame}

\begin{frame}{Before and after kyphoplasty}
    \solo{F3c.jpg} 
\end{frame}

\begin{frame}{Recurrent severe back pain}
    \begin{itemize}
        \item Recurrent severe back pain in the same area and weakness in both
              legs 3 months after kyphoplasty
        \item Refracture of the L1 vertebral body with cement fragmentation
              and neural canal encroachment
        \item L1 corpectomy involving the PMMA cement resection was performed
              for surgical decompression.
        \item VAS was 2 at 3-month F/U.
    \end{itemize}
\end{frame}

\begin{frame}{Refracture and surgical decompression}
    \solo{F3e.jpg} 
\end{frame}

\section{Discussion}
\begin{frame}{Discussion}
    \begin{itemize}
        \item The phenomenon of refracture of the cemented vertebrae after
              kypholasty has been sporadically reported in recent literatures.
        \item Only 14 cases have been reported, including the present 3 cases.
        \item Average age of 76.8 years (63--86 years)
        \item Severe osteoporosis with a mean T-score of $-3.46$ ($-5.0$ to $-3.0$).
    \end{itemize}
\end{frame}

\begin{frame}{Patient characteristics}
    \solo{T1.pdf}
\end{frame}

\begin{frame}{Patient characteristics}
    \begin{itemize}
        \item All patients obtained pain relieve immediately after kyphoplasty,
              but developed refractures of the previously operated vertebrae at
              the meantime of 3 months (1--4 months) postoperatively.
            \begin{itemize}
                \item Mechanical strength of the augmented vertebrae could not
                      be restored sufficiently by kyphoplasty in the short term.
            \end{itemize}
        \item Vast majority of OVCFs (12/14) had the presence of intravertebral
              clefts in preoperative images.
    \end{itemize}
\end{frame}

\begin{frame}{Intravertebral cleft}
    \begin{itemize}
        \item Dynamic fracture mobility related to intravertebral cleft had
              been confirmed in radiographs.
        \item Mckiernan \textit{et al} found intravertebral clefts existed in
              every mobile OVCF but were absent from every fixed OVCF.
    \end{itemize}
\end{frame}

\begin{frame}{Intravertebral cleft}
    \begin{itemize}
        \item In the treatment of OVCFs with intravertebral clefts by
              kyphoplasty, the expansion of the balloon tamp did not have the
              dramatic effect on creating a cavity within the fractured
              vertebral body but instead it mainly pushed the bony segments up
              and down, leaving the vacuum at the site of a previous
              intravertebral cleft.
        \item The injected PMMA cement therefore initially filled the limited
              cavity created by the ballooned tamp, and the more viscous
              partially cured cement still followed the pathway of least
              resistance through the intravertebral cleft.
        \item High pressure could cause the PMMA cement in that state to leak
              into the intravertebral cleft, which might easily form a solid
              lumped PMMA mass without interspersion throughout the trabeculae.
    \end{itemize}
\end{frame}

\begin{frame}{Intravertebral cleft around kyphoplasty}
    \solo{F4.jpg}
\end{frame}

\begin{frame}{Solid pattern}
    \begin{itemize}
        \item Eventually, 10 of 12 OVCFs with intravertebral cleft were filled
              with PMMA in a solid lump pattern.
        \item The solid pattern might result in insufficient filling.
        \item PMMA cement formed as a solid lump rather than as a contiguous
              bone interdigitation, would alter normal load transfer pattern of
              the cemented vertebrae so that a stress-shielding effect could
              occur in the PMMA-unaugmented bony architecture.
        \item That might result in microfracture and progressive vertebral body
              height loss, and bone--cement interface failure, even as cement
              crack.
    \end{itemize}
\end{frame}

\begin{frame}{Risk factors of refracture}
    \begin{itemize}
        \item Severe osteoporosis
        \item Intravertebral cleft 
        \item Solid pattern
    \end{itemize}
\end{frame}

\begin{frame}{Conclusions}
    \begin{itemize}
        \item Patients with OVCFs and intravertebral clefts who did not obtain
              complete pain-relief at the treated vertebral level after
              kyphoplasty should be strictly followed up.
            \begin{itemize}
                \item Especially when a solid lump injection pattern of PMMA
                      was observed in postoperative radiographs.
            \end{itemize}
        \item Attach importance to patient adherence to antiosteoporotic
              medication treatment.
    \end{itemize}
\end{frame}
\end{document}
